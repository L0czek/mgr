\cleardoublepage
\section{Modifying QEMU} \label{chap:qemu}

To enable efficient fuzzing in \textit{QEMU}, a couple of design choices must be made. First, a fast communication method between the fuzzer running beside QEMU and the test case executor inside needs to be established. This is required for various reasons, varying from passing simple messages saying that the target is ready to fetch test cases. After the communication channel is established, a compatibility layer must be implemented to integrate \textit{QEMU} with the fuzzer. For example, fuzzers like \textit{AFL} communicate via special file descriptors, shared memory regions, and signals. These file descriptors are both ends of a Linux unnamed pipe created by \textit{AFL} before launching the target. The numbers of those file descriptors are defined in \textit{AFL++} source code as constants. They are respectively 199 for communication to QEMU and 198 in the opposite direction. These are vital to the proper initialization, as it will be discussed later. The shared memory region is used to pass gathered coverage information throughout the test case execution. The main purpose of signals used by \textit{AFL} is just to kill the target if it runs too long. Since a fuzzer cannot know when a test case execution should stop, a firm timer is set. Naturally, figuring out when a test execution will end is equivalent to trying to solve the famous stop problem brought up by Alan Turing. Lastly, after each test case is executed, the target operating system might need to restore itself. It can be as simple as just rebooting it, or some more sophisticated method based on taking a snapshot of the virtual machine state before test case execution and restoring it after.

\paragraph{Chapter contents:}
This chapter begins with a description of the necessary modifications that need to be applied to the \textit{Tiny Code Generator} module. As previously discussed this enables the communication between the \textit{Fuzzer integration} module and the \textit{Test Case Decoder} running in the emulated environment. Next, it described the \textit{Fuzzer integration} module which implements the additional features required for system fuzzing. One of the most important functionalities added by it, is the ability to save and restore the state of the virtual machine. Finally, the implementation of the \textit{AFL to QEMU interface translator} is provided. This is essential to connect the open source \textit{AFL++} fuzzer to \textit{QEMU}.

\subsection{Communicating with the target Operating system}
There are many possible ways of establishing a fast communication channel between \textit{QEMU} and the target operating system. First thought might be to add a special device that the target operating system will access when needed. \textit{QEMU} allows easy creation of any number of virtual devices, from simple memory-mapped devices to sophisticated PCI ones. Unfortunately, this solution has some drawbacks, the main one being that we are using the target operating system to send the information that it has failed. A better way of accomplishing that task would be to use some special assembly instructions. Of course, no such instruction exists in the official instruction sets of real CPUs. The closest to such instruction would be the virtual machine control ones. For example, in Intel x86 architecture, a virtual machine is started using the \textit{vmenter} instruction and exists when the virtual machine executes the \textit{vmexit}. When the exit event happens, the control is passed to the virtual machines' manager, which is the \textit{Kernel Virtual Machine} module inside the Linux kernel. This module is then controlled directly by \textit{QEMU}. Sadly, running this instruction usually requires executing in CPU's kernel mode, as user mode applications can't use those instructions for security reasons. However, since \textit{QEMU} is employed, there aren't any obstacles preventing the addition of a new instruction that would perform the communication and will not have the access control the real hypervisor calls have. In the end, I have chosen to use custom \textit{hypercalls} by modifying the \textit{Tiny Code Generator} module which is responsible for emulating the virtual processor's instructions.

\subsubsection{Modyfing the \textit{Tiny Code Generator}} \label{sec:tcg}
\paragraph{Choosing the special instruction to call QEMU}
In the case of emulating ARM architecture on an Intel x86 host, the instructions are translated by the \textit{TCG} module in \textit{QEMU}. It is therefore required to modify this transpiler, to add a previously unknown instruction and its handler. First, the encoding for the newly added opcode must be found. On \textit{ARM} CPUs instructions have fixed size and the entire space is already completely allocated. Thankfully, there exists the so-called undefined instruction, which by definition guarantees to be undefined. Contrary to its name, it is utilized by various mechanisms, mostly to report errors to the operating system. When such instruction is executed, the operating system delivers a signal to the running application, which in most cases results in the termination of the process. What is interesting in that instruction, is the fact that it can take an immediate argument. The definition of this instruction can be seen in table \ref{tab:armudf}. It shows that there are 2 variants of this instruction, which allow for either 16 or 8 bit immediate parameters. As a result, choosing the encoding comes down to selecting an immediate argument that is not used by anything else inside the operating system. 

\begin{table}[h!]
    \centering
    \begin{tabular}{c|c|c}
        Opcode & Parameters & Instruction set \\
        fx\textsubscript{4}x\textsubscript{2}x\textsubscript{3}fx\textsubscript{1}e7 & 16 bit immediate - x\textsubscript{4}x\textsubscript{3}x\textsubscript{2}x\textsubscript{1} & 32 Bit instruction set \\

        x\textsubscript{1}x\textsubscript{2}de & 8 bit immediate - x\textsubscript{1}x\textsubscript{2} & 16 bit condensed so-called \textit{Thumb} mode \\
    \end{tabular}
    \caption{Definition of the UDF instruction from the Arm manual.}
    \label{tab:armudf}
\end{table}

\paragraph{Adding the instruction to the TCG}
The \textit{TCG} uses bit patterns to assign instruction's mnemonics to encodings. Then after decoding the mnemonic, the TCG parses the arguments based on the instruction type and then calls the proper function that is required to encode it using the QEMU intermediate code, so-called \textit{QEMU OPS}. Naturally, not all instructions are so simple that they are easily describable by a simple RISC-like instruction set. Therefore, the TCG module allows registering a function that has access to the entire emulated CPU state and will be called when the instruction should be executed. Such functions run inside the virtual CPU thread inside QEMU and are not restricted in what they can access as long as proper locks synchronizing QEMU structures are set. As a result, the easiest way to create a communication channel was to add a special instruction with arguments placed in the specially selected registers. Then a callback in QEMU would read those registers and perform the necessary action.

\subsection{Saving and restoring the state of the virtual machine} \label{sec:savevm}
The ability to quickly save and restore the state of the target operating system can be a very useful tool in a couple of situations. First, while executing test cases the global state of the kernel might change, which can result in poorer reproducibility of found bugs. On the other hand, any type of taking and restoring a snapshot will introduce significant delays to the system. It is therefore essential to balance the need to have good reproducibility and fuzzing speed. In detail, the amount of added delay depends solely on the chosen method of taking snapshots. For example, Tamas Lengyel in \cite{xenfuzz} showed a Linux operating system fuzzer based on \textit{Xen} hypervisor. In this solution, he used the built-in save and restore feature. Similarly, the QEMU emulator also provides a native virtual machine serialization mechanism, which will be discussed shortly.

\paragraph{Forking QEMU}
Authors of \cite{triforceafl} tried a different approach to this problem. Since QEMU is just a process running under the Linux operating system, it can be forked or cloned. This is already done and well-supported for emulating single processes inside QEMU. Sadly, this is not the case for full system emulation. The issue with the QEMU system lies in the number of threads it utilizes. Since the QEMU user is just emulating a single user thread, it also utilizes a single thread on the host machine. On the other hand, as already mentioned, the system version of QEMU uses at least 3 threads. This poses a significant obstacle, as forking a multithreaded process on a POSIX system is not supported. What will happen in such a situation is just cloning the thread that is called \textit{fork}. As a result, this is quite tricky to get working, as suddenly creating another \textit{vCPU} thread will certainly crash QEMU.

That is why the researchers designed a sophisticated procedure aimed at making this approach work. To ease out the process, it was divided into two parts. The first seen in figure \ref{fig:preforkqemu} contains all actions up to the point of the fork loop. It starts with the termination of the existing \textit{vCPU} threads, they will be restarted later in the forked thread. When this is completed, the \textit{vCPU} thread schedules a task to be executed in the context of the \textit{IO Thread}. This task is responsible for constantly spawning new \textit{IO Thread} instances by forking itself every time a test case is about to be executed. Thanks to this, each test receives the same copy of the virtual machine. The next part of this process, shown in figure \ref{fig:postfork} displays all action happening in the forked thread. The freshly cloned thread creates the new \textit{vCPU} threads using the copied virtual CPU states and then returns to its normal \textit{IO} event loop. Though this process seems rather straightforward, it is not guaranteed to work reliably in all versions of QEMU. I had severe difficulty in implementing this design to test its effectiveness. Most of the time the QEMU was crashing irregularly and I wasn't able to stabilize it. Therefore, this approach was abandoned, and other ways were explored.

\begin{figure}[h!]
    \centering

    \begin{sequencediagram}
        \newinst[2]{armcpu}{ARM CPU}
        \newthreadShift{vcpu}{vCPU}{2cm}
        \newthreadShift{iothread}{IO Thread}{2cm}

        \mess{vcpu}{Schedule fork loop}{iothread}

        \begin{call}{vcpu}{Stop CPU thread}{armcpu}{}
        \end{call}

        \begin{callself}{vcpu}{wait for io event}{}
        \end{callself}

        \begin{sdblock}{Forkserver}{}
            \begin{callself}{iothread}{endless fork loop}{}
            \end{callself}
        \end{sdblock}
    \end{sequencediagram}
    
    \caption{Pre fork actions.}
    \label{fig:preforkqemu}
\end{figure}

\begin{figure}[h!]
    \centering

    \begin{sequencediagram}
        \newinst[2]{armcpu}{ARM CPU}
        \newthreadShift{iothread}{Forked IO Thread}{2cm}

        \begin{call}{iothread}{Start new vCPU thread}{armcpu}{}
        \end{call}

        \begin{sdblock}{Continue IO tasks}{}
            \begin{callself}{iothread}{Event loop}{}
            \end{callself}
        \end{sdblock}
        
    \end{sequencediagram}
    
    \caption{Post fork actions.}
    \label{fig:postfork}
\end{figure}

\clearpage

\paragraph{QEMU's native serialization mechanism} \label{sec:qemu_nat}
Another possibility of restoring the state of the target operating system involves functionality that is already built-in \textit{QEMU}. The native snapshot mechanism is used by \textit{AIRBUS} security lab in their \textit{Gustav} fuzzer from \cite{gustavdoc}. It is primarily used to fuzz embedded \textit{Power PC} systems. The modified \textit{QEMU} that is used by their fuzzer is available at \cite{airbusqemu}. In general, this snapshot functionality relies on halting the virtual machine and then directly serializing every element the virtual machine consists of. Those entities are by default saved in the unused area of an emulated hard drive. Usually, this tool is invoked through the so-called “QEMU monitor” which is simply the user console interface of QEMU. As a result, this mechanism executes in the context of the \textit{mainloop} thread. This is specially done this way, as running the serialization requires stopping the \textit{vCPU} threads and locking the \textit{IO Thread}. Unfortunately, the special instructions used for communicating with the target are also running inside this \textit{vCPU} thread. Consequently, it is required to schedule a task to be executed in the context of \textit{mainloop}.

This process requires several actions to successfully schedule a snapshot task in the \textit{mainloop} with all the \textit{vCPU}s paused. The sequence of all steps is shown in figure \ref{fig:native_savevm}. First, the \textit{IO Thread} is locked to enable the management of other virtual processors. Then, the emulated CPUs are paused and the \textit{IO Thread} is unlocked to dispatch any remaining events. After the events processing is finished, the \textit{vCPU} threads are halted. In the end, a task is scheduled in the context of the \textit{mainloop} thread. It takes the snapshot and then resumes all virtual \textit{CPU} threads. The restore operation looks exactly like the saving one, with the only difference being that the \textit{mainloop} task restores from a snapshot instead of taking it.

\begin{figure}[h!]
    \centering

    \begin{sequencediagram}
        \newthread{mainloop}{Mainloop}
        \newinst[2]{armcpu}{ARM CPU}
        \newthreadShift{vcpu}{vCPU}{2cm}
        \newthreadShift{iothread}{IO Thread}{2cm}

        \mess{vcpu}{Lock IO Thread}{iothread}
        \begin{call}{vcpu}{Pause all vCPUs}{armcpu}{}
        \end{call}

        \mess{vcpu}{Unlock IO Thread}{iothread}

        \mess{vcpu}{Schedule vmsave}{mainloop}

        \begin{callself}{mainloop}{Take snapshot}{}
        \end{callself}

        \begin{call}{mainloop}{Resume all vCPUs}{armcpu}{}
        \end{call}
    \end{sequencediagram}

    \caption{Saving state using the native mechanism.}
    \label{fig:native_savevm}
\end{figure}

\paragraph{Creating custom serialization mechanism} \label{sec:qemu_cus}
The native snapshot mechanism does save the entire virtual machine, which in the case of a \textit{Trustzone} enabled ARM virtual machine consists of both the secure and insecure parts. However, as the fuzzing target lives solely on the secure side, saving the non-secure part is unnecessary. Furthermore, QEMU saves the snapshot on the virtual machine disk, which adds the delay of disk operations to the serialization. This can be optimized by allocating memory and copying from memory to memory the critical parts that need to be saved. In case there is no dedicated hardware that the operating system relies on, it is enough to save and restore just the following entities:
\begin{enumerate}
    \item the CPU registers,
    \item secure memory,
    \item the \textit{MPU} - the \textit{Memory Protection Unit},
    \item the \textit{MMU} - the \textit{Memory Management Unit}.
\end{enumerate}
This provides a decent speed increase and if done correctly is a reliable way of saving time. Additionally, as this entire operation can happen in the context of the \textit{vCPU} thread, no locking or callback scheduling is required. To illustrate this, the diagram of this method is shown in figure \ref{fig:custom_savevm}. As it can be seen, the only active part in this mechanism is the \textit{vCPU} thread, which does everything during the execution of the special instruction. This is much faster and easier to accomplish as long as enough of the virtual machine state is preserved.

\begin{figure}[h!]
    \centering

    \begin{sequencediagram}
        \newthread{mainloop}{Mainloop}
        \newinst[2]{armcpu}{ARM CPU}
        \newthreadShift{vcpu}{vCPU}{2cm}
        \newthreadShift{iothread}{IO Thread}{2cm}

        \begin{callself}{vcpu}{Backup vm state}{}
        \end{callself}
    \end{sequencediagram}
    
    \caption{Saving state using the custom mechanism.}
    \label{fig:custom_savevm}
\end{figure}

\paragraph{Chosen solutions}
After exhausting discussion, I decided to implement and test the following solutions:
\begin{enumerate}
    \item the \textit{QEMU's native serialization mechanism} which is later referred as the native serialization mechanism,
    \item the \textit{The custom serialization mechanism} described in the previous paragraph,
    \item fuzzing without restoring the state directly from the secure world to see how significant is the performance impact of virtual machine state preservation,
    \item fuzzing without restoring the state from the normal world to see whether the act of switching worlds adds any measurable delay.
\end{enumerate}

\clearpage
\subsection{Integration with AFLplusplus fuzzer} \label{sec:translator}
To integrate the \textit{AFLplusplus} fuzzer into the QEMU system, a couple of interfaces need to be connected. As already discussed, fuzzers from the \textit{AFL} family communicates through a shared memory region and special hard-coded pipes. In detail, the fuzzer just before starting the target creates two unnamed pipes as stated previously:
\begin{itemize}
    \item file descriptor of value \textit{198} is used to send data from the fuzzer to the target,
    \item file descriptor of value \textit{199} provides data flow in the opposite direction.
\end{itemize}
It is vital to correctly connect each of them when the fuzzed target is not a simple program but a sophisticated entity. Normally, \textit{AFL} expects that its target is compiled by the specially modified \textit{gcc} compiler that adds the required code handling all connections. In the case of fuzzing an operating system, this needs to be integrated separately into a part of the fuzzing setup, as for \textit{AFL} the target is still a normal user application. 

\paragraph{The shared memory region}
Let us start with the shared memory, as it is used just to store the coverage information. When AFL is started, it launches the target process and provides a special environment variable with the identifier of the shared memory region. This can be then used to access the region and mount it inside the target process memory space. This is everything that needs to be done with this communication channel. The \textit{AFL} will expect this memory region to be filled with data after each test case execution. The format of this data is a hash table where each element is a byte and the indexes are hashed program counter values taken from the target. To populate this table, the target operating system needs to be instrumented to increment an element each time a branch is taken. In QEMU, this can be done by modification of the \textit{TCG} module. This time, it is enough to just add a couple of instructions to the beginning of each translated code block. This is the same instrumentation that would be injected by the modified \textit{AFL's gcc} compiler. The only thing that needs to be chosen is the hash function. Commonly, it just mixes the current program counter value with the previous one after some bit-shifting operation. This proved to be a fast and efficient way of populating the coverage table. For this reason, I have chosen the hash function to be: $hash(PC_n) = (PC_n << 4) \oplus PC_{n-1}$. It similar to the one discussed previously in section \ref{sec:aflinstr}.

\paragraph{The \textit{fork server} mechanism}
To speed up the fuzzing process, \textit{AFL} implements a mechanism called the “fork server”. It is a loop in the target process that is constantly forking itself to always fuzz the copy of the program. The diagram of the inner workings of the fork server is presented in figure \ref{fig:forksrv}. In the beginning, the fork server sends AFL a message containing which features are available. It is used to, for example, specify how the test case is passed to the target. Next, the target process is allowed to finish setting itself up. This initialization step is done only once, and it is why this method is faster than just repeatedly starting the target program. Another advantage of utilizing a \textit{fork server} is the ability to choose where it will start the forking loop. This lets the target program do the initialization so that the target clones can start executing when everything is already set up. In the case of operating systems, a good point to begin the forking loop is after the devices have been initialized, and the \textit{init} program is ready to start the user shell. After the initialization phase, the fuzzing can begin. Overall, it consists of repeating the following steps:
\begin{enumerate}
    \item forking the fork server thread,
    \item sending the child's process identifier to \textit{AFL},
    \item the child process starts the actual fuzzing routine by fetching and executing the test case,
    \item calling \textit{waitpid} to check the child's exit code,
    \item send back the exit code to \textit{AFL}.
\end{enumerate}
In case when the child does not exit in the specified amount of time, \textit{AFL} uses \textit{SIGKILL} signal to terminate the target. Unfortunately, as already mentioned, I had severe issues with trying to fork the \textit{QEMU} process while running full system emulation. For this reason, a mechanism that doesn't rely on forking needs to be implemented. 

\begin{figure}[h!]
    \centering

    \begin{sequencediagram}
        \newthread{afl}{AFLplusplus}
        \newthreadShift{forkserver}{Fork server}{3cm}
        \newinst[3]{forkedth}{Forked target}{}

        \begin{sdblock}{Initialization}{}
            \mess{forkserver}{Send capabilities}{afl}
        \end{sdblock}
        
        \postlevel
        \begin{sdblock}{Fuzzing step}{}
            \begin{callself}{forkserver}{Do fork itself}{child's pid}
            \end{callself}
    
            \mess{forkserver}{Child's process id}{afl}

            \begin{messcall}{forkserver}{Cloned process}{forkedth}
                \postlevel
                \postlevel
                \postlevel
            \end{messcall}

            \prelevel
            \prelevel
            \prelevel

            \setthreadbias{center}           
            \begin{callself}{forkedth}{Start fuzzing}{Fuzzing result}
            \end{callself}
            %\begin{messcall}{forkedth}{Start fuzzing}{forkedth}
            %\end{messcall}

            \begin{call}{forkserver}{waitpid()}{forkedth}{exit code}
            \end{call}
    
            \mess{forkserver}{Send child's exit code}{afl}
        \end{sdblock}
    \end{sequencediagram}
    
    \caption{AFL fork server setup.}
    \label{fig:forksrv}
\end{figure}

\paragraph{Custom \textit{AFL} to \textit{QEMU} interface}
To utilize the interface described above with the previously discussed state saving and restoring mechanisms, it is best to keep one QEMU process and just use a communication channel to signal the start of a test case and retrieve the results. Ideally, between test case executions when AFL is performing its genetic algorithm step, all QEMU's threads should be suspended waiting for the command to fetch a new test. The easiest way to do it under the Linux operating system is to use the \textit{SIGSTOP} signal, which will halt the entire process. Then the \textit{SIGCONT} signal can be utilized to resume the execution. The only required part of the setup is a tool that will translate the fork server interface to the stop-and-resume one. 

The diagram of the designed communication can be seen in figure \ref{fig:execsrv}. It starts just like the unmodified fork server setup by sending the capabilities to \textit{AFL}, but then the similarities end. Since \textit{AFL} does not care if the supposedly newly cloned process has the same identifier as the previous one, the translator can just send the same QEMU's process identifier over and over again. Next, it waits for QEMU to finish executing the test case by waiting for the \textit{SIGSTOP} signal. In the meantime, QEMU executes the test case and after it has finished a couple of things are done by the \textit{vCPU} thread:
\begin{enumerate}
    \item the test result is sent to the translator by putting it in a variable allocated in shared memory,
    \item the \textit{SIGSTOP} signal is sent to QEMU's process to halt the virtual machine until \textit{AFL++} is ready to proceed,
    \item the \textit{vCPU} thread that is handling all the actions locks itself on a shared semaphore by executing the \textit{down()} function.
\end{enumerate}
The semaphore action is required, as sending the signal does not immediately halt the receiving process. As a result, the delay can cause a race condition where the QEMU will go forward, thinking that the next test case is ready to be executed, where the fuzzer is still waiting for the previous result. When the translator returns from \textit{waitpid} it sends the target status from the shared variable to \textit{AFL} and resumes QEMU by:
\begin{enumerate}
    \item sending \textit{SIGCONT} to QEMU's process,
    \item lifting the semaphore by executing the \textit{up()} routine to resume the operation of the \textit{vCPU} thread.
\end{enumerate}
In the end, the \textit{vCPU} thread initiates the restore from snapshot procedure if requested and proceeds with executing the next test case. By default, the test cases generated by \textit{AFL} are provided to the target via a special file, whose path is passed through the command line. Therefore, it was omitted from the diagram for clarity.

\begin{figure}[]
    \centering

    \begin{sequencediagram}
        \newthread{afl}{AFLplusplus} 
        \newthreadShift{srv}{Translator}{1.2cm}
        \newinst[1]{qemu}{QEMU}{}
        \newinst[1]{sem}{Semaphore}{}
        \newthreadShift{vcpu}{vCPU thread}{1cm}

        \begin{sdblock}{Initialization}{}
            \mess{srv}{Send capabilities}{afl}
        \end{sdblock}

        \postlevel
        \begin{sdblock}{Fuzzing step}{}
            \mess{srv}{Send QEMU PID}{afl} 
            
            \setthreadbias{west}
            \begin{call}{srv}{waitpid}{qemu}{status}

                \postlevel
                \postlevel
                \postlevel
                \postlevel
            
            \end{call}


            \prelevel
            \prelevel
            \prelevel
            \prelevel
            \prelevel

            \setthreadbias{east}
            \begin{call}{vcpu}{Start fuzzing}{qemu}{Fuzzing result}
            \end{call}

            \setthreadbias{center}
            \mess{vcpu}{target status}{srv}
            \mess{vcpu}{SIGSTOP}{qemu}

            \setthreadbias{east}
            \begin{call}{vcpu}{down()}{sem}{}
                \postlevel 
                \postlevel 
                \postlevel 
                \postlevel 
                \postlevel 
            \end{call}
            \setthreadbias{center}
            
            \prelevel
            \prelevel
            \prelevel
            \prelevel
            \prelevel
            \prelevel

            \postlevel
            \mess{srv}{target status}{afl}

            \postlevel
            \mess{srv}{SIGCONT}{qemu}

            \setthreadbias{west}
            \begin{call}{srv}{up()}{sem}{}
            \end{call}
        \end{sdblock}
    \end{sequencediagram}
    
    \caption{\textit{AFL} to \textit{QEMU} interface design.}
    \label{fig:execsrv}
\end{figure}
