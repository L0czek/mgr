\clearpage
\appendix{Running this fuzzing setup}

\paragraph{Repositories configuration} 
The fuzzing setup that this thesis describes is composed of several git repositories containing modified components taken from the default \textit{OPTEE OS} setup. For clarity, the architecture of this setup is reminded in figure \ref{fig:fuzzoverview2}. The source of the components that are crucial to the task of fuzzing can be found as follows:
\begin{itemize}
    \item the manifest configuring this repos can be viewed at \url{https://github.com/L0czek/optee_manifest/blob/master/qemu_v8.xml},
    \item \textit{AFLplusplus} is available in my fork at \url{https://github.com/L0czek/AFLplusplus},
    \item \textit{AFL to QEMU interface translator} is loaced in \url{https://github.com/L0czek/mgr/tree/master/execsrv},
    \item my modified \textit{QEMU} fork can be found in \url{https://github.com/L0czek/qemu/tree/fuzzing},
    \item the \textit{Fuzzer integration} along with the \textit{TCG} modifications can be seen in \url{https://github.com/L0czek/qemu/tree/fuzzing/fuzzer},
    \item the \textit{OPTEE OS} kernel with utilities is located at \url{https://github.com/L0czek/optee_os},
    \item the \textit{Rust} support for \textit{OPTEE OS} is provided in \url{https://github.com/L0czek/incubator-teaclave-trustzone-sdk/tree/fuzzing},
    \item the \textit{Secure services} and compiler that generated the \textit{Test Case Decoder} are available at \url{https://github.com/L0czek/incubator-teaclave-trustzone-sdk/tree/fuzzing/examples/fuzzer-rs},
    \item the \textit{Initializer} script is located at \url{https://github.com/L0czek/optee_build/blob/fuzzing/br-ext/board/qemu/overlay/bin/start.sh}.
\end{itemize}


\tikzstyle{opensource} = [rectangle, minimum width=3cm, minimum height=1cm,text centered, draw=black, fill=green!30]
\tikzstyle{custom} = [rectangle, minimum width=3cm, minimum height=1cm,text centered, draw=black, fill=orange!30]
\tikzstyle{opensourcemod} = [rectangle, minimum width=3cm, minimum height=1cm,text centered, draw=black, fill=green!60]
\tikzstyle{custommod} = [rectangle, minimum width=3cm, minimum height=1cm,text centered, draw=black, fill=orange!60]
\tikzstyle{modifiedmod} = [rectangle, minimum width=3cm, minimum height=1cm,text centered, draw=black, fill=yellow!30]
\tikzstyle{host} = [rectangle, minimum width=3cm, minimum height=1cm,text centered, draw=black, fill=brown!30]

\tikzstyle{arrow} = [thick,->,>=stealth]
\tikzstyle{darrow} = [thick,<->,>=stealth]

\pgfdeclarelayer{background0}
\pgfdeclarelayer{background05}
\pgfdeclarelayer{background1}
\pgfdeclarelayer{background2}
\pgfsetlayers{background0, background05, background1, background2, main}

\begin{figure}[h!]
    \centering

    \begin{tikzpicture}
        \node (kernel) [opensourcemod, text width=3cm] { OPTEE kernel };
        \node (target) [custommod, below of=kernel, text width=2cm, yshift=-0.5cm] { Secure services };
        \node (fuzzer) [custommod, text width=2cm, below of=target, yshift=-0.5cm] { Test case decoder };

        
        \begin{pgfonlayer}{background1}
            \node (optee) [opensource, fit={(fuzzer) (target) (kernel)}, label={OPTEE OS}] {};
        \end{pgfonlayer}

        \begin{pgfonlayer}{background05}
            \node (sec) [opensource, fit={(optee)}, text height=5cm, label={Secure world}] {};
        \end{pgfonlayer}
        
        \node (kernel2) [opensourcemod, text width=3cm, right of=kernel, xshift=4cm] { Linux kernel };
        \node (fuzz2) [custommod, below of=kernel2, yshift=-0.5cm] { Initializer };

        \begin{pgfonlayer}{background1}
            \node (buildroot) [opensource, fit={(kernel2) (fuzz2)}, label={Buildroot}] {};
        \end{pgfonlayer}

        \begin{pgfonlayer}{background05}
            \node (nor) [opensource, fit={(buildroot)}, text height=3.5cm, label={Normal world}] {};
        \end{pgfonlayer}

        \node (fuzzerint) [custommod, below of=fuzzer, yshift=-2.5cm, xshift=0cm] { Fuzzer integration };
        \node (tcg) [modifiedmod, below of=fuzz2, yshift=-4cm] { \textit{Tiny Code Generator }};

        \begin{pgfonlayer}{background05}
            \node (qemuinternals) [opensource, fit={(fuzzerint) (tcg)}, text height=1.5cm, text width=9cm, label={QEMU internal modules}] {};
        \end{pgfonlayer}

        \begin{pgfonlayer}{background0}
            \node (qemu) [opensource, fit={(optee) (fuzzerint) (tcg)}, text width=10cm, text height=10cm,label={QEMU}, xshift=0cm] {};
        \end{pgfonlayer}

        \node (gen) [opensourcemod, left of=kernel, xshift=-4.5cm] { Genetic Algorithm };
        
        \begin{pgfonlayer}{background1}
            \node (afl) [opensource, fit={(gen)}, text height=2cm, text width=4cm, label={AFL++ fuzzer}] {};
        \end{pgfonlayer}
        
        \node (srv) [custommod, below of=afl, yshift=-2cm, text width=3cm] { AFL to QEMU interface translator };

        \node (host) [host, below of=qemu, text width=14.5cm, yshift=-5cm, xshift=-2cm] { Intel x86 host computer };

        \draw [darrow] (afl) -- (srv) node[midway, left] {pipes};
        \draw [darrow] (srv) |- (fuzzerint) node[midway, above right, text width=2cm] {signals and semaphores};
        \draw [darrow] (fuzzerint) -- (fuzzer) node[midway, left] {hypercalls};
        \draw [darrow] (fuzzer) -- (target) node[midway, right] {API};
        \draw [darrow] (target) --++ (0cm, +1cm) node[midway, right] {API};
        \draw [arrow, dashed] (fuzz2) -- (target) node[midway, above] {Loads};
        \draw [arrow, dashed] (fuzz2) |- (fuzzer) node[midway, right] {Loads};
        \draw [arrow, dashed] (tcg) |- ++(-5cm, 1.8cm) node[midway, above] {Implements hypercalls};
        
    \end{tikzpicture}
    
    \caption{Fuzzing setup overview.}
    \label{fig:fuzzoverview2}
\end{figure}

\paragraph{Building and running}
Building this setup requires downloading multiple repositories using the \textit{repo} tool by providing the manifest file mentioned above. After the cloning is completed we can move to building the fuzzer. To provide a stable build environment I created a docker container with the exact versions of \textit{gcc} compiler and other tools that are required to compile \textit{OPTEE}. When the docker container is set up the build process works by executing the build commands from \textit{OPTEE} manual in the container. Naturally, the source files are mounted as a docker volume. This proved to be a reliable way of compiling everything that works independently of the host operating system version. The detailed commands that are required to set everything up are provided in listing \ref{lst:build}. 

Running this setup is done through the \textit{run.py} script located inside the project's root directory. It main goal is to prepare a command line starting the \textit{AFL++} fuzzer along with \textit{QEMU} and \textit{OPTEE OS}. All available options and switches that impact the fuzzing process can be seen in listing \ref{lst:runtool}. This program has four main modes:
\begin{enumerate}
    \item \textit{fuzzer} which as the name suggests start fuzzing using corpus provided by the \textit{--input} parameter and storing \textit{AFL++} results in \textit{--output},
    \item \textit{tcgen} runs the test case generation algorithm that traces unit tests and dumps the binary test cases to a directory provided by the next command line argument,
    \item \textit{qemu} just runs the \textit{QEMU} emulator without the fuzzer, it is usefull for debugging purposes,
    \item \textit{testcase} is a special mode that runs the \textit{Test Case Decoder} on a single test case from file or multiple from a directory.
\end{enumerate}

%\begin{minipage}{\linewidth}
\begin{lstlisting}[caption={Building the fuzzing setup}, label={lst:build}]
git clone https://github.com/L0czek/mgr
cd mgr
cd execsrv
mkdir build
cd build && cmake . && make
cd ../..
mkdir optee
cd optee
repo init -u https://github.com/L0czek/optee_manifest -m qemu_v8.xml
repo sync -j $(nproc)
cd ..
./env.sh "(cd build && make toolchains && make OPTEE_RUST_ENABLE=y CFG_TEE_RAM_VA_SIZE=0x00300000 VERBOSE=1)"
\end{lstlisting}
%\end{minipage}

%\begin{minipage}{\linewidth}
\begin{lstlisting}[caption={Fuzzing runner tool.}, label={lst:runtool}]
usage: Fuzzer runner [-h] [--afl-debug-log] [--fuzzer-debug-log] [--qemu-log-file QEMU_LOG_FILE] [--afl-log-file AFL_LOG_FILE] [--execsrv-log-file EXECSRV_LOG_FILE]
                     [--enable-statsd] [--statsd-host STATSD_HOST] [--statsd-flavor {dogstatsd,influxdb,librato,signalfx}] [--launch-terminals]
                     [--stdio-normal-port STDIO_NORMAL_PORT] [--stdio-secure-port STDIO_SECURE_PORT] [--testcase-decoding-mode {dsl,direct}] [--no-affinity]
                     [--afl-dir AFL_DIR] [--qemu-dir QEMU_DIR] [--execsrv-dir EXECSRV_DIR] [--optee-out-dir OPTEE_OUT_DIR] [--optee-build-dir OPTEE_BUILD_DIR] [--exit]
                     [--noout]
                     {fuzzer,tcgen,qemu,testcase} ...

positional arguments:
  {fuzzer,tcgen,qemu,testcase}
    fuzzer              Runner for fuzzer
    tcgen               Testcase generator
    qemu                Just run QEMU
    testcase            Run testcase

options:
  -h, --help            show this help message and exit
  --afl-debug-log       More verbose output from AFL
  --fuzzer-debug-log    More verbose log from QEMU
  --qemu-log-file QEMU_LOG_FILE
                        Store logs from QEMU to file
  --afl-log-file AFL_LOG_FILE
                        Store logs from AFL to file
  --execsrv-log-file EXECSRV_LOG_FILE
                        Store logs from execsrv to file
  --enable-statsd       Send metrics to statsd server
  --statsd-host STATSD_HOST
                        StatsD host
  --statsd-flavor {dogstatsd,influxdb,librato,signalfx}
                        Select the StatsD flavor
  --launch-terminals    Launch terminals
  --stdio-normal-port STDIO_NORMAL_PORT
                        Port to send TCP logs from normal world
  --stdio-secure-port STDIO_SECURE_PORT
                        Port to send TCP logs from secure world
  --testcase-decoding-mode {dsl,direct}
                        Select the test case decoding mechanism
  --no-affinity         Disable CPU pinning in AFL
  --afl-dir AFL_DIR     Path to AFL root dir
  --qemu-dir QEMU_DIR   Path to QEMU build directory
  --execsrv-dir EXECSRV_DIR
                        Path to exesrv build directory
  --optee-out-dir OPTEE_OUT_DIR
                        Path to optee out dir with compiled images
  --optee-build-dir OPTEE_BUILD_DIR
                        Path to optee build dir with build scripts
  --exit                Exit QEMU after tasks are done
  --noout               Supress output from afl
\end{lstlisting}
%\end{minipage}